\documentclass[times, twocolumn]{article}
\usepackage[margin=1in]{geometry}
\usepackage[T1]{fontenc}
\usepackage[utf8]{inputenc}
\usepackage{mathptmx}
\usepackage{graphicx}
\usepackage{sectsty}
\usepackage{xcolor}
\usepackage[dvipsnames]{xcolor}
\usepackage{enumitem}
\setlist{nosep}
\newcommand*{\justifyheading}{\raggedright}
\setlength{\columnsep}{0.4in}
\sectionfont{\fontsize{12}{10}\selectfont}

\renewenvironment{abstract}
 {\large
  \begin{center}
  \bfseries \abstractname\vspace{-.5em}\vspace{0pt}
  \end{center}
    \small
  \list{}{%
    \setlength{\leftmargin}{5mm}% <---------CHANGE HERE
    \setlength{\rightmargin}{\leftmargin}%
  }%
  \item\relax}
 {\endlist}

\title{\textbf{{\Large Exploring the Dynamic Between Mental Health and the Tech Industry}}}

\author{
    \normalsize
    \textbf{Brittany Stenekes} \\
    \normalsize
    Cornell University \\
    \normalsize
    {\fontfamily{pcr}\selectfont bss99@cornell.edu }
    
    \and
    \normalsize
    \normalsize
    \textbf{William Xiao} \\
    \normalsize
    Cornell University \\
    \normalsize
    {\fontfamily{pcr}\selectfont wmx2@cornell.edu }
  
    \and
    \normalsize
    \textbf{Sami Smalling} \\
    \normalsize
    Cornell University \\
    \normalsize
    {\fontfamily{pcr}\selectfont ss2676@cornell.edu }
}

\date{}

\begin{document}
\maketitle


\begin{abstract}
Mental health is a serious issue which is garnering more attention given the 
movement to virtual work and schooling due to the pandemic. In particular, the 
tech industry is known for heavily stigmatizing the subject. In this report, we 
discuss our plan to avoid over (and under-)fitting and to test the effectiveness 
of the models we develop. We also discuss the quality of our data, and display 
some results from preliminary results with their associated transformations. 
Finally, we explain what remains to be done, and how we plan to develop the 
project over the rest of the semester.
\end{abstract}

\fontsize{9.5}{9}\selectfont

\fontsize{10}{12}\selectfont

\section{Descriptive Statistics}

Data from surveys filled out in 2017-2019 were combined into a full dataset 
with 1,525 examples. The most pertinent questions were selected to go into the
cleaned dataset, which now has 29 features. These questions were about the 
respondent’s demographics, their mental health status, their comfortability 
discussing mental health, and how their workplace considers mental health 
concerns. Questions that were removed included repeated questions or ones 
relating to a previous employer. There were a few questions that were emphasized 
in bold in the survey, and these questions were in general the ones included in 
our cleaned dataset. 

We extracted basic statistics from the survey responses regarding employment 
type and environments from the combined survey results from 2017 to 2019. We 
first examined the age of survey participants, generating the following 
histograms. Since the distributions are very similar, we cannot conclude that 
age is a relevant metric to use in model fitting.

\begin{figure}[h]
    \centering
    \includegraphics[width=0.45\textwidth]{age.png}
    \caption{Age vs. Diagnosis}
    \label{fig:age}
\end{figure}

62\% of the participants are employed in a tech role, with half of all 
participants working for a tech company. About half of the tech employees-
illustrated by the light blue bars in Figure \ref{fig:respondent} have been 
diagnosed with a mental illness. Half of the participants sometimes work 
remotely, while 25\% always work remotely. Whether or not remote work contributes 
to mental health illness in the workplace is a factor we will consider in our 
further model fitting and analysis because remote work has become so prevalent 
among companies in 2020, due to COVID-19. 

\begin{figure}
    \centering
    \includegraphics[width=0.35\textwidth]{respondent_employment_env.png}
    \caption{Respondent Employment Environment.}
    \label{fig:respondent}
\end{figure}

Next, we separated the participants into two mutually exhaustive and exclusive 
groups: people who have been diagnosed with a mental illness and people who have 
not. As Figure \ref{fig:comfortability_mh} shows, people who have not been 
diagnosed with a mental illness are less comfortable talking about mental 
health issues compared to people who have been diagnosed with a disorder.  

\begin{figure}
    \centering
    \includegraphics[width=0.35\textwidth]{comfortability_mh.png}
    \caption{Comfortability of discussing mental health issues by diagnosis.}
    \label{fig:comfortability_mh}
\end{figure}

When analyzing data quality, we found that there are corrupted NaN values 
throughout the table, including in basic fields such as age, race, and gender. 
There are several optional survey questions that some did not choose to answer, 
including “Would you bring up mental health issues in an interview? Why or why not?”, 
so the entries for those rows are missing. The most critical columns have <1\% 
of the data missing. The average is around 14-50\% missing, and the most is 
around 80\%. 

There were also plenty of messy values in the gender column, including misspellings 
that were reminiscent of uncleaned text-to-speech (‘uhhhhhh genderqueer’), making 
for tens of different options that had to be cleaned down to one of ‘M’, ‘F’, and 
‘Nonbinary’. There were few enough distinct values that this could be cleaned manually. 
The rest of the columns did not need to be cleaned too heavily. 

\section{Preliminary Analysis}

We extracted demographic information about the survey participants and found 
that, of the respondents who work for a tech company, or in a tech role, about 
28\% of the respondents were female, 64\% male, and the remaining 8\% identified 
as another gender. 53\% of the female tech employees have been diagnosed with a 
mental health disorder, while 38\% of the men, and 7\% of the remaining tech 
employees have been diagnosed with a mental health disorder. These basic statistics 
motivate us to explore the possibility that females in tech are more prone to mental 
health disorders than males. Although, it could also be the case that females are 
more comfortable seeking professional mental health diagnoses than males.

We also examined workplace comfortability according to gender. 


These results are 
synthesized in table \ref{table:comfort-percentage-table}.

\definecolor{GoodGreen}{HTML}{008000}

\begin{table}
    \centering
    \begin{tabular}{lrr}
        \hline \textbf{Gender} & \textbf{Coworkers} & \textbf{Supervisor(s)} \\ \hline
        Female & \textcolor{red}{19.8\%} & 31.8\% \\
        Male & \textcolor{red}{20.8\%} & 30.4\% \\
        Other & 28.0\% & \textcolor{GoodGreen}{40.2\%} \\
        \hline
    \end{tabular}
    \caption{
        Percentage of tech employees who feel comfortable talking to other groups
        about mental health, organized by gender.
    } 
    \label{table:comfort-percentage-table} 
\end{table}

All percentages in the table above are 40\% or less, suggesting most people who 
work in the tech industry do not feel comfortable talking about mental health. 
Both males and females feel especially uncomfortable talking to their coworkers 
about their mental health concerns. 

We then performed a similar analysis of comfortability discussing mental health 
for participants who identified as White or Caucasian, Asian, Hispanic, Black, 
or African American. 60\% of people who identified as Black or African American, 
answered “Yes” to only 1 question asking about their comfort in discussing mental 
illness. About half of the people who identified as White answered “No” to all of 
the questions. There were mixed responses among people who identified as either 
Asian or Hispanic. About 3\% of these people answered “No” to all questions, while 
15\% answered “Yes” at least 75\% of the time. Based on these results illustrated 
in the plot below, it is clear that very few people are comfortable discussing 
mental health in their workplace.

\begin{figure}
    \centering
    \includegraphics[width=0.38\textwidth]{comfortability_race.png}
    \caption{Comfortability of discussing mental health issues by race.}
    \label{fig:comfortability_race}
\end{figure}


In summary, our initial data explorations generated the following ideas which will 
be useful in moving forward:

\begin{itemize}
    \item Mental illness seems common in the tech industry: About 1 in 2 tech 
    employees who participated have been diagnosed with a mental health illness. 

    \item Most people are uncomfortable discussing mental health issues during an 
    interview or with their coworkers or supervisors. 
    
    \item There is not a noticeable difference between the comfortability of males 
    versus females. However, of the participants in the tech industry, 15\% more 
    females have been diagnosed with a mental health issue when compared to males. 
    
    \item There are about the same proportions of participants with mental 
    illnesses across age groups. Age does not appear to be an important factor 
    to consider in model fitting. 
    
    \item Few people who identified as White, Black, or African American are 
    comfortable discussing mental health issues in any circumstances. 
    Comparatively, people who identified as Asian or Hispanic had more mixed 
    opinions. Therefore, we think ethnicity could potentially be useful in model 
    fitting, although we understand that survey different respondents have 
    different past experiences that are not necessarily captured by survey results.

\end{itemize}


\section{Initial Model}

The feature transformations used mirrored those we used in the AirBnB homework. 
For real-valued fields (e.g. age), we kept the real values. For named values 
(e.g. gender, country), we decided to use one-hot encodings and many-hot encodings 
for set values. We have not yet decided how to handle ordinal data. We may use 
one-hot encodings as well. For text data, we may use one of the text embedding 
tools mentioned in class to extract the word embedding vectors for the text responses. 

(Footnote): We realized while doing basic analysis that 863 out of 1,525 responses 
to the question “have you ever been diagnosed with a mental health disorder?” 
were NaN. Because questions following this question indicate that checkboxes 
were used for respondents to indicate whether or not they were diagnosed with a 
number of disorders, we assume that these NaN values correspond to respondents 
who have not been diagnosed with a mental health disorder, and categorize them 
as “no” values. 


\section{Next Steps}
Besides linear regression, we also plan to try to minimize the error with other 
metrics, including logistic and hinge loss to reduce the effect of outliers. We 
will quantify the error in our model fits using these loss functions. We will 
choose to move forward with the loss function that results in a small and similar 
testing and training error. 

Also, since there are questions in our dataset that are fairly similar, such as 
“Do you feel comfortable talking to a coworker about mental health?” and 
“Do you feel comfortable talking to a supervisor about mental health?”, we plan 
to try to use a few different regularizers. As a result, we hope to make our model 
sparse, using fewer features and making the results easy to interpret. In order 
to help prevent over and underfitting, we will make sure to use features which 
we believe to have the highest amount of correlation with the target response, 
including gender, race, workplace factors, and support from friends and family. 
This way, we can cut down on the number of features we have while still making 
sure that we still preserve the most important information. 

We will also attempt to prevent under and overfitting by using various train/test 
splitting techniques, including cross-validation by evaluating model error for 
multiple feature combinations and also bootstrapping. For the cross-validation 
procedure, we will begin by using the $\sim$10 features that we found to be 
relevant from our preliminary analysis. If our initial model training error 
is lower than our testing error, we will consider adding features about 
participants’ previous experiences with mental illness to avoid underfitting. 
Alternatively, if our initial model training error is higher than our testing 
error, we will remove features from our model. 

\end{document}
