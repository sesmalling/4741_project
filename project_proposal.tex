\documentclass[notitlepage, 11pt]{report}
\usepackage[utf8]{inputenc}
\usepackage{amsmath}
\usepackage{amssymb}
\usepackage{graphicx}
\usepackage[letterpaper, left=2cm, right=2cm, top=2cm, bottom=2cm]{geometry}
\usepackage{titlesec}
\usepackage{ctable}
\usepackage{caption}

\renewcommand{\baselinestretch}{1.5}
\renewcommand\thesection{}

\makeatletter
\newcommand*{\toccontents}{\@starttoc{toc}}
\makeatother
\renewcommand{\baselinestretch}{1.15} 
\begin{document}

\title{Identifying Mental Health Issues in the Tech Industry}
\author{Brittany Stenekes (bss99), William Xiao (wmx2), and Sami Smalling (ss2676)}
\date{}
\maketitle

\normalsize
\section{Question}
What are the greatest catalysts for mental illness in the workplace?
% Questions: 
% \begin{enumerate}
%     \itemsep-0.7em
%     \item Can we predict if someone has a mental health illness based on their employment environment?
%     \item What are the greatest warning signs that someone has or is developing a mental illness?
%     \item What are the greatest catalysts for mental illness in the workplace?
% \end{enumerate}

\section{Proposal}
Mental health is a serious issue which is garnering more attention given the 
movement to virtual work and schooling due to the pandemic. The technology 
industry more specifically is known for being male-dominated with high pressure 
to perform and compete which can make it difficult for employees to speak out
and ask for help. We believe we can provide assistance in the detection and 
awareness of mental health in the tech industry by investigating survey data 
from respondents. 

Open Source Mental Illness (OSMI) conducts a Mental Health in Tech 
Survey\textsuperscript{1} on an annual basis. In addition to demographic 
information of the participants, including age, gender, and country, the survey 
includes details about employment environments. The survey includes questions 
about discussing mental health concerns with employers, benefits and wellness 
packages, remote employment, and employee performance consequences. The 
participant’s family history of mental health, and historical mental health 
treatment that they have or have not sought is also included. This combination 
of demographic, employment, and personal response data should enable us to 
explore who is at risk for mental health complications and why. 

Investigating mental health in the tech industry could reveal current pitfalls
that exist in a company's organizational structure. Moreover, publicizing results
could de-stigmatize the conversation about mental health in the workplace and 
ideally encourage employers and employees to take mental health seriously. 

This endeavor of trying to unearth trends in mental health issues in the tech 
industry will succeed not only because of the availability and amount of data, 
but also because, as students going into technical fields ourselves, we 
understand the situation and context that cause this stigma to arise, and can 
relate more deeply to it. We are first hand witnesses of its detrimental effects, 
and have strong motivations to come up with a reliable indicator, so we can help
identify the problem at its source, and help others heal. 

\section{References}
\small
\begin{enumerate}
    \itemsep-0.7em
    \item https://www.nimh.nih.gov/health/statistics/mental-illness.shtml
    \item https://www.valuepenguin.com/average-small-business-loan-interest-rates
\end{enumerate}

\end{document}